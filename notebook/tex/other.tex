\section{Other Algorithms}
  \subsection{2SAT}
    Build the implication graph of the input by converting ORs $A \vee B$ to
    $!A \rightarrow B$ and $!B \rightarrow A$. This forms a bipartite graph.
    If there exists $X$ such that both $X$ and $!X$ are in the same strongly connected component,
    then there is no solution. Otherwise, iterate through the literals, arbitrarily assign
    a truth value to unassigned literals and propagate the values to its neighbors.
    % \code{other/two_sat.cpp}
  \subsection{DPLL Algorithm}
    A SAT solver that can solve a random 1000-variable SAT instance within a second.
    \code{other/dpll.cpp}
  \subsection{Stable Marriage}
    \ifverbose
    The Gale-Shapley algorithm for solving the stable marriage problem.
    \fi
    \code{other/stable_marriage.cpp}
  % \subsection{Algorithm X}
  %   \ifverbose
  %   An implementation of Knuth's Algorithm X, using dancing links. Solves the Exact Cover problem.
  %   \fi
  %   \code{other/algorithm_x.cpp}
  % \subsection{Matroid Intersection}
  %   Computes the maximum weight and cardinality intersection of two
  %   matroids, specified by implementing the required abstract methods, in
  %   $O(n^3(M_1+M_2))$.
  %   \code{other/matroid_intersection.cpp}
  \subsection{Cycle-Finding}
    \ifverbose
    An implementation of Floyd's Cycle-Finding algorithm.
    \fi
    \code{other/floyds_algorithm.cpp}
  \subsection{Longest Increasing Subsequence}
    \code{other/lis.cpp}
  \subsection{Dates}
    \ifverbose
    Functions to simplify date calculations.
    \fi
    \code{other/dates.cpp}
  \subsection{Simulated Annealing}
    An example use of Simulated Annealing to find a permutation of length $n$
    that maximizes $\sum_{i=1}^{n-1}|p_i - p_{i+1}|$.
    \code{other/simulated_annealing.cpp}
  \subsection{Simplex}
    \begin{verbatim}
// Two-phase simplex algorithm for solving linear programs
// of the form
//     maximize     c^T x
//     subject to   Ax <= b
//                  x >= 0
// INPUT: A -- an m x n matrix
//        b -- an m-dimensional vector
//        c -- an n-dimensional vector
//        x -- a vector where the optimal solution will be
//             stored
// OUTPUT: value of the optimal solution (infinity if
//                   unbounded above, nan if infeasible)
// To use this code, create an LPSolver object with A, b,
// and c as arguments.  Then, call Solve(x).
    \end{verbatim}
    \code{other/simplex.cpp}
  \subsection{Fast Input Reading}
    If input or output is huge, sometimes it is beneficial to optimize the
    input reading/output writing. This can be achieved by reading all input
    in at once (using fread), and then parsing it manually. Output can also
    be stored in an output buffer and then dumped once in the end (using
    fwrite). A simpler, but still effective, way to achieve speed is to use
    the following input reading method.
    \code{tricks/fast_input.cpp}
  \subsection{128-bit Integer}
    GCC has a 128-bit integer data type named \texttt{\_\_int128}. Useful
    if doing multiplication of 64-bit integers, or something needing a
    little more than 64-bits to represent. There's also
    \texttt{\_\_float128}.
  \subsection{Bit Hacks}
    \code{tricks/snoob.cpp}
